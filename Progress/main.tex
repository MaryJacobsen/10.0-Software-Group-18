\documentclass[letterpaper,10pt,draftclsnofoot,onecolumn,]{article}
\usepackage[utf8]{inputenc}
\usepackage[margin=0.75in]{geometry}
\usepackage{tikz}
\usepackage{array}
\usepackage{longtable}
\begin{document}
\noindent
\begin{titlepage}
    \begin{center}
        \vspace*{1cm}
        
        \textbf{\huge{Gymnastics Scoring Software}}
        
        \vspace{0.5cm}
        \large{The purpose of this document is to outline the progress during fall term for the proposed gymnastics scoring software.}
        
        \vspace{1.5cm}
        
        \textbf{\LARGE{10.0 Software}}
        
        \Large{Progress Document\\
        CS 461\\
        Senior Software Engineering Project\\
        Fall 2018}
        \vspace*{\fill}
        \begin{center}
            \textbf{\large{Abstract}}
        \end{center}
        \normalsize{This document outlines group 18's progress throughout fall term on the 10.0 Software project. The included sections are project purposes and goals, current project progress, problems and solutions to project progress, and retrospective.}
    \end{center}
\end{titlepage}
\section{Project Purposes and Goals}
The gymnastics meets held in Gill Coliseum are fast paced, usually televised competitions attended by thousands of fans. In order to run the competitions smoothly and provide a positive experience for fans, athletes, and staff, there must be a way to record, organize, and display the scores that the judges give to each routine using the hardware already in Gill Coliseum. The Oregon State Gymnastics Team would like new scoring software. The pre-existing software used to display and keep scores has become outdated and has issues displaying on the current hardware. Because of this, it is important for new, custom software to be built that can accurately and elegantly display scores on Gill Coliseum's hardware in a readable manner that is intuitive for fans to understand while providing all the necessary functionality such as exporting to specific formats.

\section{Current Project Progress}
Our current progress consists of the prerequisite documents as well as client verification of those documents. We have established communication with our client and met up with them several times to understand the specifications they want for the project. The documents we have completed are the problem statement, team standards, requirements document, tech review, and design document. Overall, these documents cover what the project is, what the project requires, agreements between team members, technology pieces of the project, and how the technology pieces will fit together to implement the project. In addition, by the end of the fall term, the client will have a brief executive summary of the collection of documents. The client will be able to verify the document and give our team the approval to continue with project implementation. We have yet to begin writing any code for the project. 


\section{Problems and Solutions to Project Progress}
Our project did not suffer many problems that slowed progress down substantially. Some of the biggest issues we faced was in creating our documents, specifically the tech review. This assignment would have worked well with 3 members, but with 4 members, coming up with 12 topics to cover resulted in us elaborating on insignificant systems in the project. Another issue our project presents is the large number of edge cases required. To help prevent issues with the system we will implement a unit test for each edge case to ensure the system works as expected. Lastly, some members of our group have never created a production quality system which may cause delays in some areas. 


\section{Retrospective}
% POSITIVE: Things that went well
% DELTA: Changes that need to be implemented
% ACTIONS: Specific actions that will be implemented in order to create the necessary changes
\begin{longtable}{ | p{0.225\linewidth} | p{0.225\linewidth} | p{0.225\linewidth} | p{0.225\linewidth} | }
    \hline
    Week & Positive & Delta & Actions \\
    \hline
    Week 1 & Got acquainted with the class & Not applicable & Not applicable \\
    \hline
    Week 2 & Got helpful feedback on each of our resumes. & Parts of each resume needed small changes. & Go through each resume with the feedback and make the necessary changes. \\
    \hline
    Week 3 & We all had different ideas for how to approach and solve the problem that we were given. This lead to a strong abundance of ideas and methods that we could chose in order to solve our group problem statement & We only meet up once as a group, and briefly interacted with our client which lead to a slight difficulty when writing or problem statements since we all had a vague idea of what was needed and wanted & The next week we planned a larger more formal meeting with our client and other people that knew more about tech and could help translate for our client, and started communicating more as a team.\\
    \hline
    Week 4 & We had the first meeting with the client and he explained more about the project. We combined the individual problem statements into a group problem statement. & Because we had not met with the client when we wrote the individual problem statements, content we had just learned from the client needed to be added to the group problem statement. & Plan out each week so we have the information needed to complete tasks.\\
    \hline
    Week 5 & This week was light which gave our group team to catch up and update/organize the github repo. We also started working on the requirements document & A major change that needed to be implemented was fully realizing our requirement documents and having it looked over by the client before we submitted it for grading & Planned another meeting with the Gill IT people and our client in order to have them look over our requirements document and give feedback. \\
    \hline
    Week 6 & We met up and created the team standards which contains a list of rules and agreements between team members. During this week, we also completed the requirements document. At this point, we had met with the client to discuss project requirements. & In the first draft submitted, our group forgot to include a Gantt chart. &  Added a Gantt chart.\\
    \hline

    Week 7 & The final tech reviews were completed. & Some of the tech reviews need to be edited. We had trouble finding enough tech pieces in the project for each member to have three to review so some of the pieces are small. & Each of us will look at the feedback we got and change the tech reviews accordingly. \\
    \hline
    Week 8 & We all went to the gymnastics mock meet to see how the previous system worked. & Learned what aspects of the old system that the users like \& dislike. & We planned out what features to keep and what to drop from the old system. \\
    \hline
    Week 9 & We made a general plan for the design doc and decided to meet the following week to complete it because of Thanksgiving.& We probably should have done a little more during break & We will be starting future assignments earlier\\
    \hline
    Week 10 & We completed the first draft of the design document. & After the design document was completed, we were made aware of another aspect of the design that needed to be added. The software also needs to print out score cards for the judges to write on. & We will learn more about the added requirement and update the design and requirements documents. \\
    \hline
    
\end{longtable}
\end{document}