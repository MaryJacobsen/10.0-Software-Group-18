\documentclass[a4paper, 10pt,draftclsnofoot,onecolumn]{article}\usepackage[letterpaper,margin=0.75in]
{geometry}
\usepackage[utf8]{inputenc}

\title{Problem Statement
\ \\
\ \\
CS 461 Senior Software Engineering Project
}

\author{Thomas Huynh}

\date{Fall 2018}

\usepackage{natbib}
\usepackage{graphicx}

\begin{document}



\maketitle
\section{ Abstract}
The current software used to input and display scores during meets in Gill Coliseum has dependability and compatibility issues. We will build  a new custom software solution that is specifically built for the needs and hardware that OSU (Oregon State University) is using in order to achieve optimum performance. Its function will be reading scores from multiple computers, sorting the scores, displaying on the digital score boards, modifying input after it has already been received, and exporting the scores to multiple platforms in a specified format. The agile method will be utilized which includes figuring out the requirements, designing the architecture, selecting compatible technologies, daily progress reports, and bi-weekly sprints. The end product should be able to read input scores from users, allow the scores to be edited, display ordered scores while dropping the lowest scores, and export the scores in various sorted orders.


\newpage
\section{Definition and Description}
There is currently an existing software solution used by OSU to input and display scores during gymnastics meets in Gill Coliseum. However, the sponsor has stated the current system is full of dependability and compatibility issues. The software was not built from the ground up with the available hardware in mind resulting in reliability issues and less than ideal performance. When there is an exciting gymnastics competition going on, a glitch with the software for the scoreboards can really ruin the moment for many fans and can be quite an embarrassment. Oregon State Athletics would like a new software solution with better performance and reliability.


\section{Proposed Solution}
In order to better meet Oregon State Athletics' needs, we will build a custom software solution that is built from the ground up to be compatible with the Gill Coliseum hardware. One important part of the project will be the ability to read scores from multiple score input stations since there will be several judges giving a score during meets. Then, the scores should be sorted by the software and displayed in order on the digital display boards for the audience to read. The lowest score should be dropped on the scoreboard. In case the judges make an input error when entering in a score, the input should be able to be changed after it has been received. The software should save the scoring data and export them into CSV (comma-separated values) files so the scores can be accessed later. A bonus feature for this project is an API (application programming interface) that can post live scoring data. We will use the agile method throughout the course of this project. This means our most important goal is to satisfy our client by continuously getting feedback as we report on our progress of the project. We  will satisfy our clients through daily stand ups for progress reports, bi-weekly sprints and bi-weekly meets to show our progress. Another aspect of the agile method we will implement is working closely with our client to deliver exactly what the client wants. We will gather the exact requirements from our client and design the architecture while choosing technologies that are verified to be compatible with the design. This may be a bit difficult as we do not anything about the hardware in Gill Coliseum. Another aspect of the agile method we will apply to the project is adaptability. This will require us to welcome changes that the client requests and be able to quickly change tests so that edge cases continue to be covered.

\section{Performance Metrics}
In order to call the project complete, we need to meet several performance metrics. First and most importantly, the client must be satisfied with the end product. This can be ensured by keeping the client updated regularly and frequently through progress reports. An agile principle we will follow closely is delivering short timescale sprints regularly to the client so we can get feedback through each iteration. Second, the software should be compatible with the Gill Coliseum hardware. Third, the software should be more reliable than the previous system. The exact metric to how much more reliable the new custom software should be, has not been decided yet. Fourth, the project should have optimum performance. We still need more communication with the client to know exactly what this means. Finally, the custom software should have the ability to read scores from multiple input stations, sort scores, display the scores on the boards, modify scores, and export them to CSV files. The project should be a complete product ready for use by the Oregon State gymnastics department and not just a prototype.


\end{document}
