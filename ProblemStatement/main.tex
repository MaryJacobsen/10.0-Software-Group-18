\documentclass[letterpaper,10pt,draftclsnofoot,onecolumn,]{IEEEtran}
\usepackage[utf8]{inputenc}
\usepackage[margin=0.75in]{geometry}
\usepackage{listings}
\setlength\parindent{0pt}
\begin{document}
\begin{titlepage}
    \begin{center}
        \vspace*{1cm}
        
        \textbf{Gymnastics Scoring Software}
        
        \vspace{0.5cm}
        The purpose of this paper is to define and describe a problem statement for the proposed gymnastics scoring software.
        
        \vspace{1.5cm}
        
        \textbf{Group 18}
        
        Problem Statement\\
        CS 461\\
        Senior Software Engineering Project\\
        Fall 2018
        
    \end{center}
\end{titlepage}

\begin{center}
\textbf{Project Abstract}\\
\end{center}
The gymnastics meets held in Gill Coliseum are fast paced, usually televised competitions attended by thousands of fans. In order to run the competitions smoothly and provide a positive experience for fans, athletes, and staff, there must be a way to record, organize, and display the scores that the judges give to each routine using the hardware already in Gill Coliseum. The Oregon State Gymnastics Team would like new scoring software. With the addition of a new jumbotron inside of Gill coliseum, the pre-existing software used to display and keep scores has become outdated and has issues displaying on the current hardware. Because of this, it is important for new, custom software to be built that can accurately and elegantly display scores on Gill Coliseum's hardware in a readable manner that is intuitive for fans to understand while providing all the necessary functionality such as exporting to specific formats.\\

\begin{center}
\textbf{Definition and Description}\\
\end{center}

\textbf{Background Information}
\par Oregon State Gymnastics is looking to improve the quality of experience for athletes, fans, and staff in regards to scoring at their gymnastics meets. They need a good way to get the scores from the judges and announce the official scores of athletes and teams to everyone within a couple of minutes of routines happening. Gill Coliseum holds five to six gymnastics meets per year against other PAC-12 and NCAA opponents. Oregon State Gymnastics will even be hosting the NCAA champion, UCLA this season. Because these meets are televised and accuracy is important in gymnastics, there cannot be any mistakes in scoring.\\

\textbf{Managing Scores}
\par There are multiple scores that need to be managed, the team scores, and the individual scores. The team scores are gathered by adding the top five individual scores for a team for all four events. A perfect score for a team would be 200 points. Individual scores are measured from 0 to 10 points, with 10 being a perfect score. Judges are able to deduct in increments as small as 0.05 with common scores being around 9.70 to 9.90. The individual score for a gymnast is made up of multiple scores given by judges, with the lowest and highest score being dropped if there are four or more judges, and the rest of the scores being averaged in order to make a single uniform score. Furthermore, each team can put up 6 gymnasts on each event and the top 5 on each event gets added to their team score. The scores also have to be sorted from highest to lowest for each event regardless of team for awards. And gymnasts that compete all four events must have their all around scores added up and sorted from highest to lowest regardless of team for awards.\\

\textbf{Displaying Scores}
\par Gymnastics meets at Gill coliseum require a lot of screens in order to portray information to fans, gymnasts, coaches, and staff. There are six total screens used to display the individual and team scores to the fans, and these six screens have three different aspect ratios that need to be accommodated for to make sure that the scores are easy for fans to understand. All of these screens need to be constantly updated with new information as the meet occurs. There are five scoring tables with computers one them being the master scoring table.\\

\textbf{Exporting Scores}
\par Lastly, there are multiple different outlets that the scores from the meet need to be sent to in order to be verified and catalogued for both the Oregon State Gymnastics Team, any rival gymnastics team, the PAC-12, and NCAA Gymnastics. Moreover, during the meet the scores need to be able to be printed at any time in a readable text format that can be used by coaches, staff, and judges. Also, the printable text format needs to show the score from every judge for each gymnast, and the team scores on each event to give the coaches, staff, and judges more information than the fans need.\\

\begin{center}
\textbf{Proposed Solution}\\
\end{center}

\textbf{Overview}
\par The proposed solution would be to create a program in a higher level language that handles all of the primary features stated above. One of the best ways to do this would be to create an API to handle all of the data. That way the program can use HTTP and CRUD methods to interact with the API in a secure and efficient manner. The API would be designed to handle all of the data in a single database with multiple tables, and the API will take all of the incoming data, parse it, and perform any data operations that are necessary. The program we create will essentially create CRUD methods to query the database. The program should be easy to use and allow for staff to choose what event is being scored, the gymnast, then the score from each judge; next, the API will send back a response with the event, the name of the gymnast, and her individual score. Once the last gymnast completes the event, the program will query the database to update the team score, and the newly updated team score will be sent back to the program. From here, the program will update the display boards with the new team score. The program will also continually update the display boards by team and event with every gymnast's score on the respective event.\\

\textbf{Managing Data with an API}
\par The use of an API will allow for scalability and flexibility of the program, CRUD and HTTP methods, better data management, and more security. Also, because of the API, data management can be separated from the program, and will make adding a mobile application easier.\\

\textbf{Stretch Goal}
\par Many fans want to have a mobile application that can provide live scores whether they are at the gymnastics meet or not. This stretch goal would require an Android and iOS application that could connect to the data API used in the scoring software in order to show live scores.\\

\begin{center}
\textbf{Performance Metrics}\\
\end{center}
To complete the project we must have a working prototype. The prototype will also need to be tested. Users will test it by following what would happen at a meet and it must be able to successfully complete the scoring of an entire meet. All the scores must be calculated, sorted, displayed, and exported successfully. After each event, the program must be able to print out a text document/pdf with all of the correct team and individual scores on each event and totaled so far. At the end of the meet, all of the data must be formatted in the correct way to be submitted to the Road to Nationals website.
\end{document}
        
